\documentclass[11pt]{article}

\usepackage{fullpage}
\usepackage{amsmath, amssymb, bm, cite, epsfig, psfrag}
\usepackage{graphicx}
\usepackage{float}
\usepackage{amsthm}
\usepackage{amsfonts}
\usepackage{listings}
\usepackage{cite}
\usepackage{hyperref}
\usepackage{tikz}
\usepackage{enumerate}
\usepackage[outercaption]{sidecap}
\usetikzlibrary{shapes,arrows}
%\usetikzlibrary{dsp,chains}

%\restylefloat{figure}
%\theoremstyle{plain}      \newtheorem{theorem}{Theorem}
%\theoremstyle{definition} \newtheorem{definition}{Definition}

\def\del{\partial}
\def\ds{\displaystyle}
\def\ts{\textstyle}
\def\beq{\begin{equation}}
\def\eeq{\end{equation}}
\def\beqa{\begin{eqnarray}}
\def\eeqa{\end{eqnarray}}
\def\beqan{\begin{eqnarray*}}
\def\eeqan{\end{eqnarray*}}
\def\nn{\nonumber}
\def\binomial{\mathop{\mathrm{binomial}}}
\def\half{{\ts\frac{1}{2}}}
\def\Half{{\frac{1}{2}}}
\def\N{{\mathbb{N}}}
\def\Z{{\mathbb{Z}}}
\def\Q{{\mathbb{Q}}}
\def\F{{\mathbb{F}}}
\def\R{{\mathbb{R}}}
\def\C{{\mathbb{C}}}
\def\argmin{\mathop{\mathrm{arg\,min}}}
\def\argmax{\mathop{\mathrm{arg\,max}}}
%\def\span{\mathop{\mathrm{span}}}
\def\diag{\mathop{\mathrm{diag}}}
\def\x{\times}
\def\limn{\lim_{n \rightarrow \infty}}
\def\liminfn{\liminf_{n \rightarrow \infty}}
\def\limsupn{\limsup_{n \rightarrow \infty}}
\def\GV{Guo and Verd{\'u}}
\def\MID{\,|\,}
\def\MIDD{\,;\,}

\newtheorem{proposition}{Proposition}
\newtheorem{definition}{Definition}
\newtheorem{theorem}{Theorem}
\newtheorem{lemma}{Lemma}
\newtheorem{corollary}{Corollary}
\newtheorem{assumption}{Assumption}
\newtheorem{claim}{Claim}
\def\qed{\mbox{} \hfill $\Box$}
\setlength{\unitlength}{1mm}

\def\bhat{\widehat{b}}
\def\ehat{\widehat{e}}
\def\phat{\widehat{p}}
\def\qhat{\widehat{q}}
\def\rhat{\widehat{r}}
\def\shat{\widehat{s}}
\def\uhat{\widehat{u}}
\def\ubar{\overline{u}}
\def\vhat{\widehat{v}}
\def\xhat{\widehat{x}}
\def\xbar{\overline{x}}
\def\zhat{\widehat{z}}
\def\zbar{\overline{z}}
\def\la{\leftarrow}
\def\ra{\rightarrow}
\def\MSE{\mbox{\small \sffamily MSE}}
\def\SNR{\mbox{\small \sffamily SNR}}
\def\SINR{\mbox{\small \sffamily SINR}}
\def\arr{\rightarrow}
\def\Exp{\mathbb{E}}
\def\var{\mbox{var}}
\def\Tr{\mbox{Tr}}
\def\tm1{t\! - \! 1}
\def\tp1{t\! + \! 1}

\def\Xset{{\cal X}}

\newcommand{\one}{\mathbf{1}}
\newcommand{\abf}{\mathbf{a}}
\newcommand{\bbf}{\mathbf{b}}
\newcommand{\dbf}{\mathbf{d}}
\newcommand{\ebf}{\mathbf{e}}
\newcommand{\gbf}{\mathbf{g}}
\newcommand{\hbf}{\mathbf{h}}
\newcommand{\pbf}{\mathbf{p}}
\newcommand{\pbfhat}{\widehat{\mathbf{p}}}
\newcommand{\qbf}{\mathbf{q}}
\newcommand{\qbfhat}{\widehat{\mathbf{q}}}
\newcommand{\rbf}{\mathbf{r}}
\newcommand{\rbfhat}{\widehat{\mathbf{r}}}
\newcommand{\sbf}{\mathbf{s}}
\newcommand{\sbfhat}{\widehat{\mathbf{s}}}
\newcommand{\ubf}{\mathbf{u}}
\newcommand{\ubfhat}{\widehat{\mathbf{u}}}
\newcommand{\utildebf}{\tilde{\mathbf{u}}}
\newcommand{\vbf}{\mathbf{v}}
\newcommand{\vbfhat}{\widehat{\mathbf{v}}}
\newcommand{\wbf}{\mathbf{w}}
\newcommand{\wbfhat}{\widehat{\mathbf{w}}}
\newcommand{\xbf}{\mathbf{x}}
\newcommand{\xbfhat}{\widehat{\mathbf{x}}}
\newcommand{\xbfbar}{\overline{\mathbf{x}}}
\newcommand{\ybf}{\mathbf{y}}
\newcommand{\zbf}{\mathbf{z}}
\newcommand{\zbfbar}{\overline{\mathbf{z}}}
\newcommand{\zbfhat}{\widehat{\mathbf{z}}}
\newcommand{\Ahat}{\widehat{A}}
\newcommand{\Abf}{\mathbf{A}}
\newcommand{\Bbf}{\mathbf{B}}
\newcommand{\Cbf}{\mathbf{C}}
\newcommand{\Bbfhat}{\widehat{\mathbf{B}}}
\newcommand{\Dbf}{\mathbf{D}}
\newcommand{\Gbf}{\mathbf{G}}
\newcommand{\Hbf}{\mathbf{H}}
\newcommand{\Kbf}{\mathbf{K}}
\newcommand{\Pbf}{\mathbf{P}}
\newcommand{\Phat}{\widehat{P}}
\newcommand{\Qbf}{\mathbf{Q}}
\newcommand{\Rbf}{\mathbf{R}}
\newcommand{\Rhat}{\widehat{R}}
\newcommand{\Sbf}{\mathbf{S}}
\newcommand{\Ubf}{\mathbf{U}}
\newcommand{\Vbf}{\mathbf{V}}
\newcommand{\Wbf}{\mathbf{W}}
\newcommand{\Xhat}{\widehat{X}}
\newcommand{\Xbf}{\mathbf{X}}
\newcommand{\Ybf}{\mathbf{Y}}
\newcommand{\Zbf}{\mathbf{Z}}
\newcommand{\Zhat}{\widehat{Z}}
\newcommand{\Zbfhat}{\widehat{\mathbf{Z}}}
\def\alphabf{{\boldsymbol \alpha}}
\def\betabf{{\boldsymbol \beta}}
\def\mubf{{\boldsymbol \mu}}
\def\lambdabf{{\boldsymbol \lambda}}
\def\etabf{{\boldsymbol \eta}}
\def\xibf{{\boldsymbol \xi}}
\def\taubf{{\boldsymbol \tau}}
\def\sigmahat{{\widehat{\sigma}}}
\def\thetabf{{\bm{\theta}}}
\def\thetabfhat{{\widehat{\bm{\theta}}}}
\def\thetahat{{\widehat{\theta}}}
\def\mubar{\overline{\mu}}
\def\muavg{\mu}
\def\sigbf{\bm{\sigma}}
\def\etal{\emph{et al.}}
\def\Ggothic{\mathfrak{G}}
\def\Pset{{\mathcal P}}
\newcommand{\bigCond}[2]{\bigl({#1} \!\bigm\vert\! {#2} \bigr)}
\newcommand{\BigCond}[2]{\Bigl({#1} \!\Bigm\vert\! {#2} \Bigr)}

\def\Rect{\mathop{Rect}}
\def\sinc{\mathop{sinc}}
\def\Real{\mathrm{Re}}
\def\Imag{\mathrm{Im}}
\newcommand{\bkt}[1]{{\langle #1 \rangle}}



\begin{document}

\title{Problems:  Probability and Random Processes}
\author{Prof.\ Sundeep Rangan}
\date{}

\maketitle

\begin{enumerate}

\item \emph{Functions of random variables.}
A common model for path loss between a transmitter and receiver is:
\[
    L = 32.4 + 17.3 \log_{10}(d) + 20\log_{10}(f_c)+ \xi, \quad \xi \sim {\mathcal N}(0,\sigma^2) \mbox{ [dB], }
\]
where $d$ is the distance in meters, $f_c$ is the carrier frequency
in GHz, and $\sigma=3.0$ dB. Let $G= 10^{-0.1L}$ be the channel power
gain in linear scale.
\begin{enumerate}[(a)]
\item What is $\Exp(L)$, the average path loss in dB?
\item What is $-10\log_{10}(\Exp(G))$, the path loss, averaged in linear scale?

\end{enumerate}



\item \emph{Joint density}.  Consider the joint density,
\[
    p(x,y) = xe^{-x(y+1)}, \quad x,y \geq 0.
\]
\begin{enumerate}[(a)]
  \item Find the marginal density $f(x)$.
  \item Find the conditional density $f(y|x)$?
  \item Find the conditional expectation $\Exp(Y|X=x)$.
(Hint: You could use integration by parts, or
you may use that
$\int_0^\infty x^k e^{-ax}dx = {k!}/{a^{k+1}}$
holds for any $a > 0$.)

\item Write a few lines of MATLAB code to generate 1000 samples of
$X$ and $Y$ from the given PDF.
\end{enumerate}

\item \emph{Conditional densities.}  Suppose that a receiver
sees a random number $K$ of transmitters.  
Given $K$, the received signal is,
\[
    Y \sim {\mathcal CN}(0, KE_1 + E_0),
\]
where $E_0$ is the average received energy when there are no transmitters
and $E_1$ is the additional average received energy per transmitter. 
Suppose that $K$ is Poisson distributed with mean $\bar{K}>0$.
\begin{enumerate}[(a)]
\item What is $P(|Y| \geq t|K=k)$ as a function of $t$?
\item What is $P(|Y| \geq t)$ as a function of $t$?
\item Find $E(|Y|^2)$.
\end{enumerate}

\item \emph{Random processes.}  
Let $x[n] \in \{0,1\}$ where $x[n]$ transitions from $0$ to $1$
with probability $p$ and $1$ to $0$ with probability $q$ in each time step.
Suppose that $x[n]$ is in the stationary distribution.
\begin{enumerate}[(a)]
\item Find the $\mu = \Exp(x[n])$, the stationary mean.

\item Find the $R[m] = \Exp(x[n]x[n-m])$, the autocorrelation function.
\end{enumerate}

\item \emph{Phase noise.}  Phase noise is a key challenge in receivers,
particularly at higher frequencies.  One simple model
for phase noise is as follows:  Phase noise will transform a signal $x(t)$ 
to a noisy signal $y(t)$ by the relation:
\[
    y(t) = x(t)e^{i\theta(t)}, \quad 
    \theta(t) = \int_{-\infty}^t w(u)du,
\]
where $w(u)$ is real-valued WSS Gaussian noise with 
\[
    R_w(\tau) = \sigma^2\delta(\tau), \quad 
    \sigma^2 = 2\pi f_0 10^{-0.1L},
\]
where $L$ is the phase noise level in dBc/Hz and $f_0$ is a reference frequency.
Thus, phase noise acts as time-varying phase rotation on the signal.
\begin{enumerate}[(a)]
\item What is the distribution of $\theta(t)-\theta(s)$ as a function of 
$s,t$ and $\sigma^2$.

\item What is the autocorrelation function $R_v(\tau)$ 
of the multiplicative noise factor $v(t) = e^{i\theta(t)}$?  You may
look and use the \emph{characteristic function} of a Gaussian.
    
\item Use MATLAB to plot the auto-correlation function $R_v(\tau)$ for a 
phase noise of $L= - 80$ dBc/Hz at $f_0=$ 10 kHz.  At what time $\tau$
is $R_v(\tau)=0.5 R_v(0)$?

\end{enumerate}



\end{enumerate}

\end{document}

